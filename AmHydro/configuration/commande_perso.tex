\AtBeginDocument{%                   Changer les puces des listes
	\renewcommand{\labelitemi}{\ding{228}}
	\renewcommand{\labelitemii}{\ding{227}}
	\renewcommand{\labelitemiii}{\ding{237}}
	}

\newcommand{\PointDD}[1]{\mathsf{#1} \left(E_{\mathsf{#1}}, ~  N_\mathsf{#1}\right)}
\newcommand{\pointDD}[1]{\mathsf{#1} \left(x_{\mathsf{#1}}, ~  y_\mathsf{#1}\right)}
\newcommand{\PointDDD}[1]{\mathsf{#1} \left(E_{\mathsf{#1}}, ~  N_\mathsf{#1}, ~ H_\mathsf{#1}\right)}
\newcommand{\pointDDD}[1]{\mathsf{#1} \left(x_{\mathsf{#1}}, ~  y_\mathsf{#1}, ~ z_\mathsf{#1}\right)}
\newcommand{\pt}[1]{$\mathsf{#1}$}

\newcommand{\vect}[1]{\overrightarrow{\mathsf{#1}}}% vecteur{P1}

\newcommand{\gis}[2]{\varphi_{\mathsf{#1 - #2}}}%    gis_{P1-P2}
\newcommand{\dir}[1]{r_\mathsf{#1}}%                 r_{P1}
\newcommand{\dis}[2]{d_{\mathsf{#1 - #2}}}%          d_{P1-P2}

\newcommand{\E}[1]{E_\mathsf{#1}}%                   E_P1
\newcommand{\N}[1]{N_\mathsf{#1}}%                   N_P1
\newcommand{\y}[1]{y_\mathsf{#1}}%                   y_P1
\newcommand{\x}[1]{x_\mathsf{#1}}%                   x_P1

\newcommand{\ex}[1]{\vect{e_x^{#1}}}%                Vecteur orthonormal e_x
\newcommand{\ey}[1]{\vect{e_y^{#1}}}%                Vecteur orthonormal e_y                        
\newcommand{\ez}[1]{\vect{e_z^{#1}}}%                Vecteur orthonormal e_z

\newcommand{\mat}[1]{\mathbf{#1}}%                   Matrice

\newcommand{\onglet}[1]{{\color{red} \textit{#1}}}
\newcommand{\bouton}[1]{{\color{blue} \textbf{#1}}}

\newcommand{\X}{{\color{red} X}}
\newcommand{\W}{{\color{darkgreen} WWWW}}
\newcommand{\A}{{\color{red} AA}}
\newcommand{\DMY}{{\color{red} AAAAMMJJ}}
\newcommand{\DBX}{{\color{blue} DBX}}

\newcommand\Warning{%              Panneau attention avec couleur
 \makebox[1.4em][c]{%
 \makebox[0pt][c]{\raisebox{.1em}{\small!}}%
 \makebox[0pt][c]{\color{red}\Large$\bigtriangleup$}}}%


%Avec babel, nous avons :
%\og et \fg nous permettent d’obtenir les guillemets français « et »  ;
%\no, \nos, \No et \Nos nous permettent d’obtenir « n° » et ses variantes majuscules et plurielles  ;
%\ier, \iere, \ieme, \iers, \ieres et \iemes nous permettent d’obtenir « er », « re » et autres  ;
%\primo, \secundo, \tertio et \quarto pour « 1° », « 2° », « 3° » et « 4° » et la commande \frenchenumerate avec un nombre en paramètre nous permet d’obtenir les autres nombres.

\setlength\parindent{0pt}%       Taille de l'alinéa 
\setlength{\fboxsep}{2mm}%       définir l'écart entre le texte et l'encadré
\setlength{\fboxrule}{0.1mm} %     définir l'épaisseur du trait de l'encadré


\newcommand{\formule}[1]{{\color{purple} #1}}
\newcommand{\graphique}[1]{{\color{orange} #1}}
\chapter{Analyse de séries de données de débits}

\section{Explication}

\section{Séries annuelles, avec débits maximaux}
\fbox{ %fbox est utilisé pour voir les bords de la minipage
    \begin{minipage}[l]{17cm}
        L'étude est la marche à suivre conviennet pour des séries statistiques avec un débit maximal annuel ! \\
        Cela veut dire que pour chaque année (et chaque mois) nous avons le débit maximal, le tout sur une période donnée (plusieurs années) (ex. Tab. \ref{tab:serieAnnuelleMaximum})
    \end{minipage}
}

\begin{table}[H]
    \centering
    \begin{tabular}{|c||c|c|c|c|c|c|c|c|c|c|c|c|}
        \hline
        \textbf{Année} & \textbf{Jan} & \textbf{Fev} & \textbf{Mar} & \textbf{Avr} & \textbf{Mai} & \textbf{Juin} & \textbf{Jui} & \textbf{Aoû} & \textbf{Sep} & \textbf{Oct} & \textbf{Nov} & \textbf{Dec} \\
        \hline \hline
        \textbf{1965}  & 11 & \cellcolor{green}10 & 14 & 15 & 160 & 205 & 205 & \cellcolor{red}350 & 145 &  84 &  21 & 18 \\
        \hline
        \textbf{1966}  & \cellcolor{green}17 & 19 & 17 & 47 & 105 & \cellcolor{red}175 & 155 & 150 &  97 & 125 &  25 & 20 \\
        \hline
        \textbf{1967}  & \cellcolor{green}17 & 19 & 20 & 39 & 145 & \cellcolor{red}320 & 240 & 210 & 110 &  75 &  38 & 35 \\
        \hline
        \textbf{1968}  & 19 & \cellcolor{green}15 & 21 & 53 & 125 & 205 & \cellcolor{red}220 & 115 & 140 &  57 & 185 & 40 \\
        \hline
        \textbf{1969}  & 15 & \cellcolor{green}13 & 14 & 32 & 120 & \cellcolor{red}205 & 190 & 175 &  82 &  65 &  45 & 22 \\
        \hline
        \textbf{\dots} &    &    &    &    &     &     &     &     &       &     &     &      \\
        \hline
        \textbf{1992}  & 14 & \cellcolor{green}13 & 17 & 62 & 110 & \cellcolor{red}290 & 225 & 215 & 175 &  75 &  46 & 38 \\
        \hline
        \textbf{1993}  & 28 & 42 & 38 & 49 & 125 & 200 & 180 & 150 & \cellcolor{red}460 & 170 &  37 & \cellcolor{green}27 \\
        \hline
    \end{tabular}
    \caption{Tableau avec les débits maximums pour chaque mois entre les années 1965 et 1993}
    \label{tab:serieAnnuelleMaximum}
\end{table}

\subsection{Procédure pour déterminer et extrapoler les temps de retour}
\begin{enumerate}
    \item \textbf{\underline{Vérification de la stationnarité des données statistiques :}} \\
    \begin{itemize}
        \item Tracer le graphique des débits maximum par années comme la Fig XX
        \item Vérification que cela ne varie par en fonction des années (courbe de tendance)
        \item Visualiser l'évolution des crues de pointe en fonction des années donne un bon aperçu d'une dérive quelconque
        \item \Warning Si les données ne sont pas stationnaires ; cela ne sert à rien de continuer la procédure pour déterminer les débits extrapolés
    \end{itemize}
    \bigskip
    \item \textbf{\underline{Vérification de l'homogénéité des données statistiques :}}
    \begin{itemize}
        \item Tracer le graphique des débits maximum par années comme la Fig XX
        \item Vérification optionnelle (car implique d'avoir les débits maximaux mensuels)
        \item Vérification que cela ne varie par en fonction des années (courbe de tendance)
        \item Visualiser l'évolution des crues de pointe en fonction des années donne un bon aperçu d'une dérive quelconque
    \end{itemize}
    \bigskip
    \item \textbf{\underline{Calcul des temps de retour $T$ :}}
    \begin{enumerate}
        \item Classer les débits par ordre décroissant (du plus grand au plus petit)
        \item Inscrire le rang de chaque débit
        \item Calculer le temps de retour selon la formule choisie (cf. Tab. \ref{tab:formuleTempsRetour}) \\
        Conseil : utiliser la \formule{\underline{formule de Hazen}} \\
    \end{enumerate}
    \bigskip
    \item \textbf{\underline{Calcul des paramètres de la loi de Gumbel :}} \\
    \textit{Appelé aussi ajustement statistiques} 
    \begin{enumerate}
        \item Calcul de la fonction $F(Q_{obs})$
        \item Calcul des divers paramètres des données statistiques :
        \begin{itemize}
            \item Moyenne des débits observés \formule{$\bar{Q}_{obs} = \Sigma Q_{obs_{i}}$}
            \item Ecart-type de la moyenne \formule{$\sigma_{Q_{obs}}$} (Excel : \texttt{ECARTYPE.STANDARD()})
            \item Paramètre \formule{$a = \bar{Q}_{obs} - 0.5772 \cdot b$}
            \item Paramètre \formule{$b = \cfrac{\sqrt{6}}{\pi} \cdot \sigma_{Q_{obs}}$}
        \end{itemize}
        \item Calcul du débit Gumbel :
        \begin{enumerate}
            \item \formule{$U = -\ln \left( -\ln \left( F(Q_{obs}) \right) \right)$}
            \item \formule{$Q_{Gumbel} = a + b\cdot U$}
        \end{enumerate}
        \item \graphique{Créer le graphique \textbf{GRAPH01} avec les éléments suivants :}
        \begin{itemize}
            \item \graphique{Abscisse : $U$ (variable réduite de la loi de Gumbel)} \\
            \graphique{Échelle logarithmique}
            \item \graphique{Ordonnée : $Débit  \;  \text{[m3/s]}$}
            \item \graphique{Données : débits annuels mesurés / débits Gumbel}
        \end{itemize}
        \item \graphique{Créer le graphique \textbf{GRAPH02} avec les éléments suivants :}
        \begin{itemize}
            \item \graphique{Abscisse : $Temps \; retour  \; [années]$} \\
            \graphique{Échelle logarithmique}
            \item \graphique{Ordonnée : $Débit  \;  \text{[m3/s]}$}
            \item \graphique{Données : débits annuels mesurés / débits Gumbel}
        \end{itemize}
    \end{enumerate}
    \bigskip
    \item \textbf{\underline{Extrapolation d'un débit en fonction du temps de retour :}}
    \begin{enumerate}
        \item Reprendre les paramètres \formule{$a$} et \formule{$b$} déterminer plus tôt
        \item Fixer les temps de retour \formule{$T_{extrapolé}$} souhaités (5, 10, 20, 30, 50, 100, 300 ans)
        \item Calcul de \formule{$F(Q) = 1-\cfrac{1}{T_{extrapolé}}$}
        \item Calcul de \formule{$U = -\ln \left( -\ln \left( F(Q) \right) \right)$}
        \item Calcul de \formule{$Q_{extrapolé} = a + b \cdot U$}
        \item Ajouter la donnée \formule{{$Q_{extrapolé}$}} sur les \graphique{\textbf{GRAPH01}} et \graphique{\textbf{GRAPH02}}
    \end{enumerate}
\end{enumerate}




\section{Séries gonflées}
Une série gonflée est une série de données statistiques où nous avons \underline{2 ou plus débits maximaux par année}.

\section{Séries tronquées}
Une série tronquée est une série de données statistiques où les \underline{débits sont supérieurs à $Q_\text{seuil}$.} \\
\Warning Si le seuil est trop bas, on prend des débits très fréquents et des débits extrêmes ; qui ne sont peut-être pas homogène. \\
On prend les séries tronquées pour obtenir les débits fréquents de temps de retour faible; voire inférieur au temps de retour années. \\
Privilégiez les séries tronquées aux séries gonflées.
\chapter{Analyse de séries de données de débits}

\section{Explication}
\begin{itemize}
    \item \underline{\textbf{Temps de retour moyens :}} 2 à 5 ans
    \item \underline{\textbf{Temps de retour rares :}} 10, 30, 100, 300 ans, \dots ! Cela dépend surtout des objectifs de protection.
\end{itemize}

\section{Séries annuelles, avec débits maximaux}


\begin{table}[H]
    \centering
    \begin{tabular}{|c||c|c|c|c|c|c|c|c|c|c|c|c|}
        \hline
        \textbf{Année} & \textbf{Jan} & \textbf{Fev} & \textbf{Mar} & \textbf{Avr} & \textbf{Mai} & \textbf{Juin} & \textbf{Jui} & \textbf{Aoû} & \textbf{Sep} & \textbf{Oct} & \textbf{Nov} & \textbf{Dec} \\
        \hline \hline
        \textbf{1965}  & 11 & \cellcolor{green}10 & 14 & 15 & 160 & 205 & 205 & \cellcolor{red}350 & 145 &  84 &  21 & 18 \\
        \hline
        \textbf{1966}  & \cellcolor{green}17 & 19 & 17 & 47 & 105 & \cellcolor{red}175 & 155 & 150 &  97 & 125 &  25 & 20 \\
        \hline
        \textbf{1967}  & \cellcolor{green}17 & 19 & 20 & 39 & 145 & \cellcolor{red}320 & 240 & 210 & 110 &  75 &  38 & 35 \\
        \hline
        \textbf{1968}  & 19 & \cellcolor{green}15 & 21 & 53 & 125 & 205 & \cellcolor{red}220 & 115 & 140 &  57 & 185 & 40 \\
        \hline
        \textbf{1969}  & 15 & \cellcolor{green}13 & 14 & 32 & 120 & \cellcolor{red}205 & 190 & 175 &  82 &  65 &  45 & 22 \\
        \hline
        \textbf{\dots} &    &    &    &    &     &     &     &     &       &     &     &      \\
        \hline
        \textbf{1992}  & 14 & \cellcolor{green}13 & 17 & 62 & 110 & \cellcolor{red}290 & 225 & 215 & 175 &  75 &  46 & 38 \\
        \hline
        \textbf{1993}  & 28 & 42 & 38 & 49 & 125 & 200 & 180 & 150 & \cellcolor{red}460 & 170 &  37 & \cellcolor{green}27 \\
        \hline
    \end{tabular}
    \caption{Tableau avec les débits maximums pour chaque mois entre les années 1965 et 1993}
    \label{tab:serieAnnuelleMaximum}
\end{table}

\subsection{Procédure pour déterminer et extrapoler les temps de retour}
\begin{enumerate}
    \item \textbf{\underline{Vérification de la stationnarité des données statistiques :}} \\
    \begin{itemize}
        \item Tracer le graphique des débits maximum par années comme la Fig XX
        \item Vérification que cela ne varie par en fonction des années (courbe de tendance)
        \item Visualiser l'évolution des crues de pointe en fonction des années donne un bon aperçu d'une dérive quelconque
        \item \Warning Si les données ne sont pas stationnaires ; cela ne sert à rien de continuer la procédure pour déterminer les débits extrapolés
    \end{itemize}
    \bigskip
    \item \textbf{\underline{Vérification de l'homogénéité des données statistiques :}}
    \begin{itemize}
        \item Tracer le graphique des débits maximum par années comme la Fig XX
        \item Vérification optionnelle (car implique d'avoir les débits maximaux mensuels)
        \item Vérification que cela ne varie par en fonction des années (courbe de tendance)
        \item Visualiser l'évolution des crues de pointe en fonction des années donne un bon aperçu d'une dérive quelconque
    \end{itemize}
\end{enumerate}

\section{Séries gonflées}
Une série gonflée est une série de données statistiques où nous avons \underline{2 ou plus débits maximaux par année}.

\section{Séries tronquées}
Une série tronquée est une série de données statistiques où les \underline{débits sont supérieurs à $Q_\text{seuil}$.} \\
\Warning Si le seuil est trop bas, on prend des débits très fréquents et des débits extrêmes ; qui ne sont peut-être pas homogène. \\
On prend les séries tronquées pour obtenir les débits fréquents de temps de retour faible; voire inférieur au temps de retour années. \\
Privilégiez les séries tronquées aux séries gonflées.
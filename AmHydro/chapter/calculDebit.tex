\chapter{Calculs et prévisions des débits en fonction du temps de retour}

\section{Explication}

\section{Séries annuelles, avec débits maximaux}
\fbox{ %fbox est utilisé pour voir les bords de la minipage
    \begin{minipage}[l]{17cm}
        L'étude est la marche à suivre conviennet pour des séries statistiques avec un débit maximal annuel ! \\
        Cela veut dire que pour chaque année (et chaque mois) nous avons le débit maximal, le tout sur une période donnée (plusieurs années) (ex. Tab. \ref{tab:serieAnnuelleMaximum})
    \end{minipage}
}

\begin{table}[H]
    \centering
    \begin{tabular}{|c||c|c|c|c|c|c|c|c|c|c|c|c|}
        \hline
        \textbf{Année} & \textbf{Jan} & \textbf{Fev} & \textbf{Mar} & \textbf{Avr} & \textbf{Mai} & \textbf{Juin} & \textbf{Jui} & \textbf{Aoû} & \textbf{Sep} & \textbf{Oct} & \textbf{Nov} & \textbf{Dec} \\
        \hline \hline
        \textbf{1965}  & 11.3 &  9.7 & 14.0 & 14.5 & 160 & 205 & 205 & \cellcolor{red}350 & 145.0 &  84 &  21 & 17.5 \\
        \hline
        \textbf{1966}  & 16.6 & 18.5 & 16.6 & 47.0 & 105 & \cellcolor{red}175 & 155 & 150 &  97.0 & 125 &  25 & 19.5 \\
        \hline
        \textbf{1967}  & 16.7 & 18.8 & 20.0 & 39.0 & 145 & \cellcolor{red}320 & 240 & 210 & 110.0 &  75 &  38 & 35.0 \\
        \hline
        \textbf{1968}  & 19.2 & 14.6 & 21.0 & 53.0 & 125 & 205 & \cellcolor{red}220 & 115 & 140.0 &  57 & 185 & 40.0 \\
        \hline
        \textbf{1969}  & 14.8 & 13.4 & 13.9 & 32.0 & 120 & \cellcolor{red}205 & 190 & 175 &  82.0 &  65 &  45 & 22.0 \\
        \hline
        \textbf{\dots} &      &      &      &      &     &     &     &     &       &     &     &      \\
        \hline
        \textbf{1992}  & 13.5 & 12.5 & 16.7 & 62.0 & 110 & \cellcolor{red}290 & 225 & 215 & 175.0 &  75 &  46 & 38.0 \\
        \hline
        \textbf{1993}  & 28.0 & 42.0 & 38.0 & 49.0 & 125 & 200 & 180 & 150 & \cellcolor{red}460.0 & 170 &  37 & 27.0 \\
        \hline
    \end{tabular}
    \caption{Tableau avec les débits maximums pour chaque mois entre les années 1965 et 1993}
    \label{tab:serieAnnuelleMaximum}
\end{table}

\subsection{Procédure pour déterminer et extrapoler les temps de retour}
\begin{enumerate}
    \item \textbf{\underline{Vérification de la stationnarité des données statistiques :}} \\
    \begin{itemize}
        \item Graphique des débits maximum par années
        \item Vérification que cela ne varie par en fonction des années (courbe de tendance)
        \item Visualiser l'évolution des crues de pointe en fonction des années donne un bon aperçu d'une dérive quelconque
        \item \Warning Si les données ne sont pas stationnaires ; cela ne sert à rien de continuer la procédure pour déterminer les débits extrapolés
    \end{itemize}
    \bigskip
    \item \textbf{\underline{Vérification de l'homogénéité des données statistiques :}}
    \begin{itemize}
        \item Vérification optionnelle (car implique d'avoir les débits maximaux mensuels)
        \item Vérification que cela ne varie par en fonction des années (courbe de tendance)
        \item Visualiser l'évolution des crues de pointe en fonction des années donne un bon aperçu d'une dérive quelconque
    \end{itemize}
    \bigskip
    \item \textbf{\underline{Calcul des temps de retour $T$ :}}
    \item \textbf{\underline{Calcul des paramètres de la loi de Gumbel :}}
    \item \textbf{\underline{Extrapolation d'un débit en fonction du temps de retour :}}
\end{enumerate}


\section{Séries gonflées}
\section{Séries tronquées}
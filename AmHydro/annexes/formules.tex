\chapter{Formules}

\section{Conversion volumes et débits}
\begin{table}[h!]
    \centering
    \begin{tabular}{|c|c|c|c|c|c|c|c|c|c|c|c|}
        \hline
        \multicolumn{3}{|c|}{\textbf{$m^3$}} & \multicolumn{3}{|c|}{\textbf{$dm^3$}} & \multicolumn{3}{|c|}{\textbf{$cm^3$}} & \multicolumn{3}{|c|}{\textbf{$mm^3$}} \\
        \hline
         & &                                 & $hL$ & $daL$ & $L$                    & $dL$ & $cL$ & $mL$                    & & & \\
        \hline \hline
         & &                               1 & 0    &   0   & 0                      & & &                                   & & & \\  
        \hline
         & &                               0. & 0    &   0   & 1                     & & &                                   & & & \\
        \hline
    \end{tabular}
\end{table}

\begin{align*}
    \MS{1} &= \LS{1000}   \\
           &= \MH{3.6e3} \\
           &= \LH{3.6e6} \\
\end{align*}

\section{Temps de retour}
\begin{table}[H]
    \centering
    \begin{tabular}{ccc}
        \toprule
        \textbf{Nom} & \textbf{Formule}            & \textbf{Notes}     \\
        \toprule
        Weibull      & $\cfrac{n+1}{r}$            & Utilisée aux USA   \\
        \midrule
        Médiane      & $\cfrac{n+0.365}{r-0.3175}$ &                    \\
        \midrule
        Hosking      & $\cfrac{n}{r-0.35}$         &                    \\
        \midrule
        Blom         & $\cfrac{n+0.25}{r-0.375}$   &                    \\
        \midrule
        Cunnane      & $\cfrac{n+0.20}{r-0.40}$    &                    \\
        \midrule
        Gringorten   & $\cfrac{n+0.12}{r-0.44}$    &                    \\
        \arrayrulecolor{red} \midrule
        Hazen        & $\cfrac{n}{r-0.5}$          & Utilisée en France \\
        \bottomrule
    \end{tabular}
    \caption{Différentes formules de calculs des temps de retour. $n$ est le nombre d'années total de l'étude; $r$ est le rang}
    \label{tab:formuleTempsRetour}
\end{table}

\section{Loi de Gumbel -- Séries annuelles}
\begin{table}[H]
    \centering
    \begin{tabular}{c|cc|c}
        \# & \textbf{Paramètres}                & \textbf{Formules}                                                                                                                         & \textbf{Commentaires}      \\
        \hline
        1  & $\overline{Q}_\text{mes}$          & $\overline{Q}_\text{mes} = \cfrac{1}{n} \cdot \displaystyle{\sum_{i=0}^{n} Q_i}$                                                          & Moyenne des débits mesurés \\
        \hline
        2  & $\sigma_{\overline{Q}_\text{mes}}$ & $\sigma_{\overline{Q}_\text{mes}} = \sqrt{\cfrac{1}{n} \cdot \displaystyle{\sum_{i=0}^{n} \left(Q_i - \overline{Q}_\text{mes}\right)^2}}$ & Ecart-type de la moyenne des débits mesurés \\
        \hline
        3  & $a$                                & $a = \overline{Q}_\text{mes} - 0.5772 \cdot b$                                                                                            & \\
        \hline
        4  & $b$                                & $b = \cfrac{\sqrt{6}}{\pi} \cdot \sigma_{\overline{Q}_\text{mes}}$                                                                        & \\
        \hline
        5  & $F(Q)$                             & $F(Q) = 1-\cfrac{1}{T}$                                                                                                                   & \\
        \hdashline                   
        6  & $F(Q)$                             & $F(Q) = e^{-e^{\frac{- \left(Q-a\right)}{b}}}$                                                                                            & \\
        \hline
        7  & $Q$                                & $Q = a + b \cdot U$                                                                                                                       & Débit selon la loi de Gumbel \\
        \hline
        8  & $U$                                & $U = -\ln \left[ -\ln \left(F(Q)\right)\right]$                                                                                          & Variable réduite de Gumbel   \\
    \end{tabular}
    \caption{Ajustement statistique par la loi de Gumbel}
    \label{tab:loiGumbel}
\end{table}

\section{Loi de Gumbel -- Séries tronquées}
\begin{table}[H]
    \centering
    \begin{tabular}{c|cc|c}
        \# & \textbf{Paramètres}                & \textbf{Formules}                                                                                                                         & \textbf{Commentaires}      \\
        \hline
        1  & $\overline{Q}_\text{mes}$          & $\overline{Q}_\text{mes} = \cfrac{1}{n} \cdot \displaystyle{\sum_{i=0}^{n} Q_i}$                                                          & Moyenne des débits mesurés \\
        \hline
        2  & $\sigma_{\overline{Q}_\text{mes}}$ & $\sigma_{\overline{Q}_\text{mes}} = \sqrt{\cfrac{1}{n} \cdot \displaystyle{\sum_{i=0}^{n} \left(Q_i - \overline{Q}_\text{mes}\right)^2}}$ & Ecart-type de la moyenne des débits mesurés \\
        \hline
        3  & $a_\text{exp}$                     & $a_\text{exp} = \overline{Q}_\text{mes} - b_\text{exp}$                                                                                   & \\
        \hline
        4  & $b_\text{exp}$                     & $b_\text{exp} = \sigma_{\overline{Q}_\text{mes}}$                                                                                         & \\
        \hline
        5  & $\lambda$                          & $\lambda = \cfrac{\text{nombre de débits}}{\text{nombre de valeurs}}$                                                                     & \\                                               
        \hline
        6  & $a$                                & $a = a_\text{exp} + b_\text{exp} \cdot \ln \left(\lambda\right)$                                                                          & \\                                                                                         
        \hline
        7  & $F(Q)$                             & $F(Q) = 1-\cfrac{1}{T}$                                                                                                                   & \\
        \hdashline                   
        8  & $F(Q)$                             & $F(Q) = e^{-e^{\frac{- \left(Q-a\right)}{b}}}$                                                                                            & \\
        \hline
        9  & $Q$                                & $Q = a + b \cdot U$                                                                                                                       & Débit selon la loi de Gumbel \\
        \hline
        10 & $U$                                & $U = -\ln \left[ -\ln \left(F(Q)\right)\right]$                                                                                          & Variable réduite de Gumbel   \\
    \end{tabular}
    \caption{Ajustement statistique par la loi exponentielle et la loi de Gumbel}
    \label{tab:loiGumbelAjuste}
\end{table}